\documentclass{jarticle}
\usepackage{ketpic2e,ketlayer2e,graphicx,color}
\usepackage{amsmath,amssymb}
\usepackage{emath,emathEy,emathMw} %% emathのコマンドを使う
\usepackage[c]{EMesvect}% ベクトル用の矢印
\def\tasukikata{2}% emathのタスキがけの型を選択
\usepackage{cancel}%% 約分の斜線を入れるため
\usepackage[dvipdfmx]{pict2e}
\usepackage{okumacro}
%\usepackage{pagecolor}\newpagecolor{red}
\input{(kubo)OvalArrow24a.txt}

\usepackage[dvipsnames,svgnames,x11names]{xcolor}

\setmargin{30}{20}{20}{20}

\begin{document}

\Large
kettaskで使われている色を確認する。\\
先頭が実際の数値指定、右に近い色を名前で指定して列挙\\

\fcolorbox[rgb]{0,0,0}{1.00,1.00,0.00}{[1.00,1.00,0.00]}:
\fcolorbox{black}{Yellow}{Yellow},
\fcolorbox{black}{yellow}{yellow}\\

\fcolorbox[rgb]{0,0,0}{0.74,0.95,1.00}{[0.74,0.95,1.00]}:
\fcolorbox{black}{PaleTurquoise}{PaleTurquoise},
\fcolorbox{black}{Cyan}{Cyan}\\

\fcolorbox[rgb]{0,0,0}{1.00,0.79,0.85}{[1.00,0.79,0.85]}:
\fcolorbox{black}{Pink}{Pink}\\

\fcolorbox[rgb]{0,0,0}{0.90,1.00,0.70}{[0.90,1.00,0.70]}:
\fcolorbox{black}{PaleGreen}{PaleGreen},
\fcolorbox{black}{GreenYellow}{GreenYellow},
\fcolorbox{black}{YellowGreen}{YellowGreen}\\

\fcolorbox[rgb]{0,0,0}{0.00,1.00,0.00}{[0.00,1.00,0.00]}:
\fcolorbox{black}{green}{green},
\fcolorbox{black}{Lime}{Lime},
\fcolorbox{black}{Green}{Green}\\

\fcolorbox[rgb]{0,0,0}{1.00,0.00,0.00}{[1.00,0.00,0.00]}:
\fcolorbox{black}{red}{red}\\

\fcolorbox[rgb]{0,0,0}{1.00,0.75,0.45}{[1.00,0.75,0.45]}:
\fcolorbox{black}{Apricot}{Apricot},
\fcolorbox{black}{Orange}{Orange},
\fcolorbox{black}{orange}{orange},
\fcolorbox{black}{LightSalmon}{LightSalmon},\\
\fcolorbox{black}{Goldenrod1}{Goldenrod1},
\fcolorbox{black}{Peach}{Peach},
\fcolorbox{black}{PeachPuff}{PeachPuff}\\


※ xcolorで複数のオプションの色を指定するため次のようにした。\\
\verb|\usepackage[dvipsnames,svgnames,x11names]{xcolor}|\\
colorを使った後でも使える。xcolorを使った後にオプション付きのxcolorは,エラーが出るので,初めに指定するときにオプションを指定する。

大文字と小文字を区別する。\\

\end{document}
